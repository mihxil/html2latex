
% HTML
\documentstyle[esperant,html]{article} %twocolum, a4wide
\title{Fromagxo}
\author{Willem Elsschot}
\date{}

\def\cxapitro#1{\section{#1}}

\def\rrim#1{}
\def\rim#1{}
%\def\rim#1{\footnote{#1}}

\def\a#1{(#1)}
\def\ax#1#2{(#2)}


\begin{document}
\maketitle
\begin{rawhtml}
<hr>
<ul>
<li> Elnederlandigita de Michiel Meeuwissen.
<li> versio: 1a de oktobro 1999
<li><!--XXX--><!--a href="./fromagxo.html"-->iksa<!--/a-->
 kaj 
   <!--888--><a href="./8/fromagxo.html">unikoda</a> versio
<li><a href="./fromagxo.ps.gz">fromagxo.ps.gz</a>: PostSkripta versio (pakita per gzip p. 123 kb
(60 pagxoj))<br>
<li><a href="../elsschot/esp.html">Informo
pri Elsschot </a>
<li><a href="../esperanto.html"> Esperanta
hejmpagxo </a>
<li> Mi numeris la alineojn por faciligi viajn komentojn. 
<li> Petu al mi se vi bezonas alian formon.
</ul>
<center>
nombrilo: <!--#exec cgi="/cgi-bin/counter/5"-->
<a href="http://www.nedstat.nl/cgi-bin/viewstat?name=fromagxo"><img
src="http://www.nedstat.nl/cgi-bin/nedstat.gif?name=fromagxo" border=0 alt="" 
width=32 height=32></a>
</center>
<h1>Enhavo</h1>
\end{rawhtml}
\input{fromtks.fun}
\end{document}
























